%% start of file `template.tex'.
%% Copyright 2006-2013 Xavier Danaux (xdanaux@gmail.com).
%
% This work may be distributed and/or modified under the
% conditions of the LaTeX Project Public License version 1.3c,
% available at http://www.latex-project.org/lppl/.
% % % Requiere texlive-bibtex-extra
\documentclass[11pt,a4paper,sans]{moderncv}
\moderncvstyle{classic} % style options are 'casual' (default), 'classic', 'oldstyle' and 'banking'
\moderncvcolor{green}
\usepackage[utf8]{inputenc}
\usepackage[scale=0.75]{geometry}
%\setlength{\hintscolumnwidth}{3cm} % if you want to change the width of the column with the dates
%\setlength{\makecvtitlenamewidth}{10cm} % for the 'classic' style, if you want to force the width allocated to your name and avoid line breaks. be careful though, the length is normally calculated to avoid any overlap with your personal info; use this at your own typographical risks...
% personal data
\name{Guillermo}{Olmedo}
%\address{San Martín 3853}{Luján de Cuyo, M5507EVY}{Mendoza, Argentina}
\phone[mobile]{+54~(261)~677~9074}
\phone[fixed]{+54~(261)~496~3020 \textit{ext} 252}
\email{olmedo.guillermo@inta.gob.ar}
\social[github]{midraed} % optional, remove / comment the line if not wanted
%\photo[64pt][0.4pt]{picture} % optional, remove / comment the line if not wanted; '64pt' is the height the picture must be resized to, 0.4pt is the thickness of the frame around it (put it to 0pt for no frame) and 'picture' is the name of the picture file
% to show numerical labels in the bibliography (default is to show no labels); only useful if you make citations in your resume
\makeatletter
%\renewcommand*{\bibliographyitemlabel}{\@biblabel{\arabic{enumiv}}}
\makeatother
%\renewcommand*{\bibliographyitemlabel}{[\arabic{enumiv}]}% CONSIDER REPLACING THE ABOVE BY THIS
% bibliography with mutiple entries
\usepackage{multibib}  % Hay que correr bibtex sobre el aux de cada grupo a mano!!
\newcites{inwork,book,conferences,others}{{Árticulos con referato aceptados},{Libros, Capítulos de libros},{Trabajos en congresos, simposios y seminarios}, {Informes Tecnicos, Publicaciones no seriadas}}
%----------------------------------------------------------------------------------
% content
%----------------------------------------------------------------------------------

%% FF: Separar el documento en bloques, que luego agrego o quito a partir
%% de usos de \input{seccion}


\begin{document}
%\begin{CJK*}{UTF8}{gbsn} % to typeset your resume in Chinese using CJK
%----- resume ---------------------------------------------------------
\makecvtitle
\section{Educación}
\cventry{2001--2008}{Ingeniero Agrónomo}{Universidad Nacional de Cuyo}{Mendoza, Argentina}{\textit{}}{}  % arguments 3 to 6 can be left empty
\cventry{1995--2000}{Bachiller Agrotécnico y Enólogo}{Liceo Agrícola y Enológico}{Mendoza, Argentina}{\textit{}}{}

\section{Formación de posgrado}
\cvitem{Título}{\emph{Doctorado en Agronomía} en curso}
\cvitem{Comité}{\textit{Ph.D.} S. Ortega Farías; \textit{M.Sc.} R. Vallone; \textit{Ph.D.} H. Vila; \textit{Ph.D.} M. Balzarini}
\cvitem{Tema}{Estimación de la evapotranspiración real de vid a diferentes escalas mediante modelos matemáticos y sensores remotos.}

\section{Experiencia}
\subsection{Gestión}
\cventry{2014--2016}{Coordinador Técnico de Módulo}{Programa Nacional de Suelos}{INTA}{}{Proyecto Nacional de bases y nuevas herramientas para la cartografía de suelos.\newline{}%
Objetivos Específicos:%
\begin{itemize}%
\item Proveer nueva información cartográfica de suelos en distintas regiones del país y que constituya una base para el desarrollo territorial y el estudio de procesos vinculados al recurso;
  \begin{itemize}%
  \item Ajustar metodologías de MDS para elaborar mapas de suelos y variables edáficas;
  \item Capacitar a los profesionales y técnicos en diferentes temáticas necesarias para el Relevamiento y Cartografía de Suelos;
  \end{itemize}
\end{itemize}}
\cventry{2014--2016}{Presidente}{Comisión Científica de Geografía de Suelos}{Asociación Argentina de la Ciencia de Suelos}{}{Organización Conferencia: \textit{Ph.D.} Alfred Zinck. Tema: Suelos, Información, Sociedad.\\ Facultad de Agronomía. UBA.}
\cventry{2011--fecha}{Responsable}{Laboratorio de Geomática}{INTA EEA Mendoza}{}{}
\subsection{Investigación}
\cventry{2012  a la fecha}{Investigador}{Instituto Nacional de Tecnología Agropecuaria}{EEA Mendoza}{}{\newline{}
Proyectos Cartera 2013--2019:
\begin{itemize}
\item PRET Atención a las problemáticas de los nuevos modelos productivos del Valle de Uco.
\item PE Bases conceptuales y nuevas herramientas para la cartografía de suelos.
\item PRET Desarrollo territorial del oasis norte de Mendoza.
\item PE Gestión del agua y el riego para el desarrollo sostenible de los territorios.
\item I Gestión integral de la información de suelos para la planificación y uso sostenible del recurso.
\item PE Herramientas metodológicas para la gestión de la información de suelos y la evaluación de tierras.
\item PE Necesidades de agua de los cultivos y estrategias de riego.
\item PE Soporte técnico y capacitación en procesos de ordenamiento territorial rural.
\item PE Tecnología de riego para diferentes sistemas productivos.
\end{itemize}    
Carteras Anteriores:
\begin{itemize}
\item PE Aplicación de métodos para el ordenamiento territorial rural
\item PR Apoyo al desarrollo vitivinícola regional
\item PR Apoyo para mejorar el uso y gestión de los recursos naturales y el ordenamiento del territorio rural.
\item PE Cartografía Digital de Suelos: Implementación y Validación de nuevas tecnologías para el relevamiento de suelos
	\item PE Desarrollo de tecnologías para la optimización del riego
\item PE Dinámica de la cobertura y uso de las tierras\end{itemize}}
\cventry{2009--2012}{Becario de prácticas profesionales}{Instituto Nacional de Tecnología Agropecuaria}{EEA Mendoza}{}{Teledetección aplicada al relevamiento de los Recursos Naturales - Laboratorio de Geomática.}
\cventry{2008--2009}{Becario Alumno}{Instituto Nacional de Tecnología Agropecuaria}{EEA Mendoza}{}{Apoyo para mejorar el uso y gestión de los Recursos Naturales}
\cventry{2005--2009}{Contratado}{Instituto Nacional de Tecnología Agropecuaria}{EEA Mendoza}{}{Apoyo para mejorar el uso y gestión de los Recursos Naturales\\
Mapa de Aptitud de Suelos con Fines de Riego y de Riesgo de Contaminación Edáfica de los Oasis Irrigados de la Provincia de Mendoza. 
Ajuste de la Tecnología de Agricultura de Precisión a la Fruticultura.}
\cventry{2004--2005}{Auxiliar de 2da}{Cátedra de Edafología. Facultad de Ciencias Agrarias}{Universidad Nacional de Cuyo}{}{Proyecto: Ajuste de Metodologías de Cuantificación de Variación Espacial de Características Edáficas a Distintas Escalas.}
\subsection{Docencia}
\cventry{1 al 3 de octubre 2014}{Módulo: Relación Agua Suelo Planta}{Docente}{Maestría de Riego y Drenaje}{Universidad Nacional de Cuyo}{Instrumental de suelo y planta para riego. Lisimetría. Balance hídrico y de energía. \textit{23 horas en el ciclo 2014}}
\cventry{5 de junio 2014}{Sistemas de información en el Agro}{Docente}{Maestría Binacional de Diseño de Sistemas Electrónicos aplicados a la Agronomía}{Universidad Nacional de San Luis}{Herramientas de geomática para el estudio de los 
recursos naturales (suelo, agua, planta). \textit{8 horas en el ciclo 2014}}
\cventry{16 al 20 de diciembre 2013}{Curso Internacional de Mapeo Digital de Suelos.}{Docente}{Centro de Investigación Agrícola Tropical, Santa Cruz, Bolivia}{Organizado por la Organización de las Naciones Unidas para la Agricultura y Alimentación}{\textit{40  horas}}
\cventry{2012--2013}{Introducción a la cartografía digital de suelos.}{Docente}{EMBRAPA Solos, Río de Janeiro, Brasil}{Organizado por la Organización de las Naciones Unidas para la Agricultura y Alimentación}{Modelado de funciones en profundidad, Redes neuronales artificiales, árboles de decisión.\\ \textit{20  horas presencial 25 al 29 de febrero 2013\\
20 horas online\\
20 horas presencial 24 al 28 de setiembre 2012}}
\cventry{2011--2012}{Módulo: Zonificación Vitícola y Agricultura de precisión}{Docente y Coordinador}{Maestría de Viticultura y Enología}{Universidad Nacional de Cuyo}
{Teledectección, sistemas de infomarción geográfica, geoestadística, viticultura de precisión, GPS, sensores proximales. \textit{30  horas. Ciclos 2011, 2012}}
\cventry{14 al 18 de mayo de 2012}{Curso de Introducción al mapeo digital de suelos}{Docente}{}{Proyecto Nacional INTA “Cartografía Digital de Suelos: Implementación y Validación de nuevas tecnologías para el relevamiento de suelos”}
{Introducción a R, Estadística, Análisis de regresión, geoestadística, Regression-Kriging y validación.\textit{30  horas. \textit{20 horas}}}
\cventry{2009--2010}{Colaboración en otros módulos}{Docente}{Maestría de Viticultura y Enología}{Universidad Nacional de Cuyo}
{Modelos matemáticos y sensores. Módulo: Relación Agua Suelo Planta. \textit{5 horas. Ciclo 2010}\\
Técnicas geoespaciales para el mapeo de variables edáficas y de cultivo. Módulo: Suelos y Riego. \textit{5 horas. Ciclo 2009}}
\section{Publicaciones} %papersinwork, book, conferences, others
\nociteinwork{Bedendo2015, Angueira2015}
\bibliographystyleinwork{abbrv}
\bibliographyinwork{inwork}                   % 'publications' is the name of a BibTeX file
\nocitebook{Gardi2014, Olmedo2014}
\bibliographystylebook{abbrv}
\bibliographybook{inbooks}                   % 'publications' is the name of a BibTeX file

\nociteconferences{Vila2007,Vallone2008,Olmedo2009a,Vallone2010,Vallone2011,Olmedo2011,Vallone2011a,Olmedo2012,Olmedo2012a,Maffei2012,Olmedo2013,MansillaBaca2013,Olmedo2014VP,Vallone}
\bibliographystyleconferences{abbrv}
\bibliographyconferences{conferences}                   % 'publications' is the name of a BibTeX file
\nociteothers{Olmedo2009, Villagra2010, Olmedo2010, Vallone2014}
\bibliographystyleothers{abbrv}
\bibliographyothers{others}                   % 'publications' is the name of a BibTeX file
\section{Becas y premios}
\cventry{2008--2009}{Beca para estudiantes}{INTA EEA Mendoza}{Mendoza, Argentina}{}{Tema: Apoyo para mejorar el uso y gestión de los recursos naturales}
\cventry{2009--2012}{Beca para prácticas profesionales}{INTA EEA Mendoza}{Mendoza, Argentina}{}{Teledetección aplicada al relevamiento de los recursos naturales}
\cventry{mayo de 2015}{Full Scholarship}{Ministerio de Comercio de la República Popular China}{Nanjing, China}{}{Seminar on Applications of Information Technology on Agriculture}
\section{Afiliaciones}
\cvitem{Socio}{Asociación Argentina de la Ciencia de Suelos}
\cvitem{Miembro}{Open Source Geospatial Foundation}
\cvitem{Miembro}{Foundation for Open Access Statistics}
\section{Revisión de trabajos}
\cvitemwithcomment{2014}{Revista Argentina de la Ciencia del Suelo}{Asociación Argentina de la Ciencia de Suelo}
\cvitemwithcomment{2014}{Agrociencia}{Facultad de Agronomía de la Universidad de la República; Instituto Nacional de Investigación Agropecuaria, Uruguay.}
\section{Idiomas}
\cvitem{Español}{lengua nativa}
\cvitem{Inglés}{Intermediate Listener and Speaker, Advanced Reading and Writing}
\section{Intereses}
\cvitem{Suelos}{Cartografía, Génesis, Clasificación}
\cvitem{Agua}{Eficiencia, Necesidades, Evapotranspiración}
\section{Dirección de Recursos Humanos}
\subsection{Trabajos de Tesis}
\cventry{2013}{Asesor}{Leandro Javier Cara}{Tesis para optar al título de Lic. en Geología}{Universidad de San Juan, Argentina}{Tema: Análisis Geomorfológico Digital del Valle Intermontano del Departamento de Tupungato Provincia de Mendoza.}
\cventry{2014}{Asesor}{María Cristina Angueira}{Tesis para optar por el título de Doctor}{Universidad de Córdoba, España}{Tema: Relevamiento de suelos aplicando nuevas técnicas de Geomática: un caso Santiago del Estero, Argentina.}
\subsection{Pasantías}
\cventry{2013}{Nicolás Suvieta.}{Tec. En Geomática}{}{3 meses}{Evolución de la superficie agrícola del Valle de Uco en 1983-2013, mediante sensores remotos.}
\cventry{2012}{Ayelén Massó}{Geógrafa Prof.}{}{3 meses}{Sistema de información Territorial del área de influencia de la EEA INTA Mendoza.}
\cventry{2011}{Romina Hernandez}{Ing. Agrónomo}{}{3 meses}{Caracterización de suelos para la instalación de un lisímetro de pesada}
\cventry{2010}{Federico Albrieu}{Ing. Agrónomo}{}{3 meses}{Uso de sondas de inducción electromagnética para la caracterización de suelos}
\cventry{2009}{María Sol Villegas}{Ing. Agrónomo}{}{3 meses}{Uso de sondas de inducción electromagnética para la caracterización de suelos}
\cventry{2009}{Martín Retamales}{Ing. Agrónomo}{}{3 meses}{Uso de sondas de inducción electromagnética para la caracterización de suelos}
\end{document}
