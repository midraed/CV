\documentclass[10pt]{article}
\usepackage{array, xcolor, lipsum, bibentry}
\usepackage[margin=3cm]{geometry}
\usepackage[utf8]{inputenc}
\usepackage[spanish]{babel} %sudo apt-get install texlive-lang-spanish 
\usepackage[utf8]{inputenc}
\usepackage{natbib}

\title{\bfseries Guillermo Federico Olmedo}
\author{}
\date{}
 
\definecolor{lightgray}{gray}{0.8}
\newcolumntype{L}{>{\raggedleft}p{0.14\textwidth}}
\newcolumntype{R}{p{0.8\textwidth}}
\newcommand\VRule{\color{lightgray}\vrule width 0.5pt}

 
\begin{document}
\maketitle
\vspace{1em}
\begin{minipage}[ht]{0.48\textwidth}
\textbf{Personal}\\
Rondeau 421\\
Ciudad, M5500CCI\\
Mendoza, Argentina\\
+54 261 6779074\\
\textit{guillermo.olmedo@gmail.com}
\end{minipage}
\begin{minipage}[ht]{0.48\textwidth}
\textbf{Laboral}\\
San Martín 3853\\
Luján de Cuyo, M5507EVY\\
Mendoza, Argentina\\
+54 261 4963020 (ext 252)\\
\textit{olmedo.guillermo@inta.gob.ar}
\end{minipage}
\vspace{20pt}
 
 \section*{Educación}
 \begin{tabular}{L!{\VRule}R}
 2014 a la fecha&{Estudiante de PhD en: \textit{Estimación de la evapotranspiración real de vid a diferentes escalas mediante modelos matemáticos y sensores remotos}, Universidad Nacional de Cuyo}\\[5pt]
 2001--2008&Ingeniero Agrónomo, Universidad Nacional de Cuyo, Argentina\\
 \end{tabular} 
 
  
\section*{Experiencia en Investigación}
\begin{tabular}{L!{\VRule}R}
2012 a la&{\bf Investigador}, Instituto Nacional de Tecnología Agropecuaria\\
fecha& Área de Recursos Naturales, EEA Mendoza\\
& Coordinador Técnico de Módulo: \textit{Mapeo Digital de Suelos}. Programa Nacional de Suelos (2014--2016)\\\multicolumn{2}{c}{}\\
2009--2012&{\bf Becario de Prácticas Profesionales}, Instituto Nacional de Tecnología Agropecuaria\\
& Área de Recursos Naturales, EEA Mendoza\\
& Proyecto Nacional: 
\end{tabular}
 

 
\section*{Idiomas}
\begin{tabular}{L!{\VRule}R}
Klingon&Mother tongue\\
{\bf English}&{\bf Fluent}\\
French&Fluent (DELF 2010)\\
Japanese&Fair\\
\end{tabular}
 
\bibliographystyle{plain}
\bibliography{publicaciones}


 
\section*{Publicaciones}
\begin{tabular}{L!{\VRule}R}
2006&\bibentry{}\\[5pt]
1986&\bibentry{}\\
\end{tabular}


{\vspace{20pt}\newline\newline
\vspace{20pt}
\scriptsize\hfill {Última actualización: \today}
 \end{document}